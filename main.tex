\documentclass[10pt,a4paper]{xresume}

% Change the font if you want to.
%\geometry{left=1cm,right=9cm,marginparwidth=6.8cm,marginparsep=1.2cm,top=1cm,bottom=1cm}

\definecolor{VividPurple}{HTML}{2E64FE}
\definecolor{SlateGrey}{HTML}{2E2E2E}
\definecolor{LightGrey}{HTML}{666666}
\colorlet{heading}{VividPurple}
\colorlet{accent}{VividPurple}
\colorlet{emphasis}{SlateGrey}
\colorlet{body}{LightGrey}

\name{张三}
\tagline{DBA \& Developer}
\photo{2.5cm}{me}

\personalinfo{
  \location{四川 \textbullet{} 成都}
  \phone{199-1234-5678}
  \email{japinli@hotmail.com}\\
  % Uncomment the following if you are female.
  % \female{}
  \birth{1993-11-29}
  \github{github.com/japinli}
  \blog{\href{https://blog.japinli.top}{blog.japinli.top}}
}

\begin{document}

\begin{adjustwidth}{}{}
  \makecvheader{}
\end{adjustwidth}

\divider\vspace{-0.5cm}

\begin{minipage}[t]{0.3\textwidth}
  \cvsection{\faCogs{}专业技能}
  \cvskill{C}{5}
  \cvskill{C++}{3}
  \cvskill{Go}{2}
  \cvskill{Python}{3}
  \cvskill{SQL}{4}
  \cvskill{Shell}{4}
  \cvskill{\LaTeX{}}{3}

  \divider\smallskip

  \cvtag{PostgreSQL}
  \cvtag{Greenplum}
  \cvtag{GCC}
  \cvtag{Git}
  \cvtag{GDB}
  \cvtag{Linux}
  \cvtag{MacOS}
  \cvtag{Windows}
  \cvtag{Emacs}
  \cvtag{Vim}

  \cvsection{\faCertificate{}证书 \& 奖项}
  \cvtag{Bilibili大学优秀毕业生}
  \cvtag{英语六级}

  \cvsection{\faGraduationCap{}教育背景}
  \cvevent{计算机技术 \textbullet{} 硕士}{Bilibili大学}{2014.09 \textendash{} 2017.06}{}
  \divider{}
  \cvevent{土木工程 \textbullet{} 学士}{家里蹲大学}{2010.09 \textendash{} 2014.06}{}

  \cvsection{\faLanguage{}语言}
  \cvskill{中文}{5}
  \cvskill{英语}{4}
\end{minipage}
% Do not keep a blank line here.
\hfill%
% Do not keep a blank line here.
\begin{minipage}[t]{0.66\textwidth}

  \cvsection{\faClockO{}工作经历}

  \cvjob{高级软件工程师}{XXX科技有限公司}{2021.03 \textendash{} 现在}{四川\textbullet{}成都}
  \begin{itemize}
  \item 负责XXX的设计与开发
  \item 负责XXX的开发
  \end{itemize}

  \divider{}

  \cvjob{软件工程师}{XXX技术有限公司}{2019.12 \textendash{} 2021.02}{四川\textbullet{}成都}
  \begin{itemize}
  \item 负责XXX的开发
  \item 实现了XXX功能
  \end{itemize}

  \divider{}

  \cvjob{实习软件开发}{BBB技术有限公司}{2018.10 \textendash{} 2019.02}{四川\textbullet{}成都}
  \begin{itemize}
  \item 负责XXX的开发
  \item 完成了XXX工作
  \end{itemize}

  \divider{}

  \cvjob{实习软件开发}{BBB技术有限公司}{2017.07 \textendash{} 2018.10}{四川\textbullet{}成都}
  \begin{itemize}
  \item 负责XXX的开发
  \item 完成了XXX工作
  \end{itemize}

  \cvsection{\faGithubAlt{}开源贡献}
  \cvopensource{\href{https://git.postgresql.org/gitweb/?p=postgresql.git&a=search&h=HEAD&st=commit&s=japin}{PostgreSQL}}
  \begin{itemize}
  \item \href{https://git.postgresql.org/gitweb/?p=postgresql.git;a=commit;h=641f3dffcdf1c7378cfb94c98b6642793181d6db}{恢复 get\_constraint\_index() 函数语义}
  \item \href{https://git.postgresql.org/gitweb/?p=postgresql.git;a=commit;h=bd74c4037c4ee268db46e983bcc0f1e0a9f7ab72}{修复自定义参数不能以下划线开始的问题}
  \item \href{https://git.postgresql.org/gitweb/?p=postgresql.git;a=commit;h=89f059bdf52cc9a86b890d42ceed92237123479e}{移除 BeginCopyFrom() 中不必要的内存上下文切换}
  \item \href{https://git.postgresql.org/gitweb/?p=postgresql.git;a=commit;h=f3d4019da5d026f2c3fe5bd258becf6fbb6b4673}{修复逻辑复制中日期和float不一致导致的复制异常}
  \item \href{https://git.postgresql.org/gitweb/?p=postgresql.git;a=commit;h=82ed7748b710e3ddce3f7ebc74af80fe4869492f}{添加 ALTER SUBSCRIPTION ... ADD/DROP ... 语法}
  \item \href{https://git.postgresql.org/gitweb/?p=postgresql.git;a=commit;h=5c0f7cc5442108e113d4fb88c952329b467e2c6a}{修复 contrib/auto\_explain 中的内存泄露}
  \item \href{https://git.postgresql.org/gitweb/?p=postgresql.git;a=commit;h=40ab64c1ec1cb9bd73695f519cf66ddbb97d8144}{修复逻辑复制删除表后刷新异常的问题}
  \item \href{https://git.postgresql.org/gitweb/?p=postgresql.git;a=commit;h=9877374bef76ef03923f6aa8b955f2dbcbe6c2c7}{添加 idle\_session\_timeout 功能}
  \end{itemize}

\end{minipage}

\newpage

\cvsection{\faGithubAlt{}开源贡献}

\cvopensource{
  \href{https://github.com/greenplum-db/gpdb/search?q=japinli&type=issues}{Greenplum} \&
  \href{https://github.com/greenplum-db/gporca/search?q=japinli&type=issues}{gporca} \&
  \href{https://github.com/greenplum-db/gpbackup/search?q=japinli&type=issues}{gpbackup}}
\begin{itemize}
\item \href{https://github.com/greenplum-db/gpdb/commit/75283bc70e9783f9197cd141464941f5bbf09967}{修复 gpdb 中 pgrowlocks 未初始化变量导致的编译错误}
\item \href{https://github.com/greenplum-db/gporca/commit/6258ef0743ef685d9f8fc529a629fb92eb512289}{提供 gporca 在 aarch64 平台上的支持}
\item \href{https://github.com/greenplum-db/gporca/commit/54ce8b3c0de2c63a5ea8eee8c2ce5d513214eba7}{修复 gporca 在 Solaris 平台上的编译错误}
\item \href{https://github.com/greenplum-db/gpbackup/commit/e28df4dfcf6f208134c8be69c22258220c333471}{修复 gpbackup 在 Solaris 平台上由于 syscall 导致的测试问题}
\item \href{https://github.com/greenplum-db/gpbackup/commit/fa25bf640e3ad80f2d2be3aca775bf238c627d8f}{修复 gpbackup 单元测试清理问题}
\item \href{https://github.com/greenplum-db/gpbackup/commit/2843de46e509b8485ce11095d73d7ea7964ed390}{修复测试程序中 GOPATH 路径问题}
\end{itemize}

\cvsection{\faCubes{}项目经验}

\cvproject{\href{https://github.com/japinli/pg_node2graph}{pg\_node2graph}}
          {个人项目}
          {C++, Graphviz}{2022.08 \textendash{} 现在}
\begin{itemize}
\item 将 PostgreSQL 的字符节点树转化成图片
\end{itemize}

\divider{}

\cvproject{\href{https://github.com/japinli/xresume}{基于 \LaTeX{} 的简历模版}}
          {个人项目}
          {\LaTeX{},MacOS}{2021.10 \textendash{} 现在}
\begin{itemize}
\item 基于 \href{https://www.overleaf.com/articles/grant-morgans-first-latex-resume/jtdbtcctgnrr}{Grant Morgan} 和 \href{https://www.overleaf.com/articles/peter-rasmussens-resume-data-scientist/bphkfprrcnwv}{Peter Rasmussen} 的 \LaTeX{} 模版二次开发
\item 增加 \verb|\cvjob|,\verb|\cvproject|,\verb|\cvopensource| 等命令来细化各个部分的输出
\item 采用 Peter Rasmussen 简历模版中的方式对页面进行分栏
\item 添加中文支持
\end{itemize}

\end{document}
