\documentclass[10pt,a4paper]{xresume}

% Change the font if you want to.
%\geometry{left=1cm,right=9cm,marginparwidth=6.8cm,marginparsep=1.2cm,top=1cm,bottom=1cm}

\definecolor{VividPurple}{HTML}{2E64FE}
\definecolor{SlateGrey}{HTML}{2E2E2E}
\definecolor{LightGrey}{HTML}{666666}
\colorlet{heading}{VividPurple}
\colorlet{accent}{VividPurple}
\colorlet{emphasis}{SlateGrey}
\colorlet{body}{LightGrey}

\name{张三}
\tagline{DBA \& Developer}
\photo{2.5cm}{me}

\personalinfo{
  \location{四川 \textbullet{} 成都}
  \phone{199-1234-5678}
  \email{sangzhang@hotmail.com}\\
  % Uncomment the following if you are female.
  % \female{}
  \birth{1993-11-29}
  \github{github.com/sanzhang}
  \blog{\href{https://www.baidu.com/s?wd=sangzhang}{www.sanzhang.com}}
}

\begin{document}

\begin{adjustwidth}{}{}
  \makecvheader{}
\end{adjustwidth}

\divider\vspace{-0.5cm}

\begin{minipage}[t]{0.3\textwidth}
  \cvsection{\faCogs{}专业技能}
  \cvskill{C}{5}
  \cvskill{C++}{3}
  \cvskill{Go}{2}
  \cvskill{Python}{3}
  \cvskill{SQL}{4}
  \cvskill{Shell}{4}
  \cvskill{\LaTeX{}}{3}

  \divider\smallskip
  
  \cvtag{PostgreSQL}
  \cvtag{Greenplum}
  \cvtag{GCC}
  \cvtag{Git}
  \cvtag{GDB}
  \cvtag{Linux}
  \cvtag{MacOS}
  \cvtag{Windows}
  \cvtag{Emacs}
  \cvtag{Vim}

  \cvsection{\faCertificate{}证书 \& 奖项}
  \cvtag{Bilibili大学优秀毕业生}
  \cvtag{英语六级}

  \cvsection{\faGraduationCap{}教育背景}
  \cvevent{计算机技术 \textbullet{} 硕士}{Bilibili大学}{2014.09 \textendash{} 2017.06}{}
  \divider{}
  \cvevent{土木工程 \textbullet{} 学士}{家里蹲大学}{2010.09 \textendash{} 2014.06}{}
  
  \cvsection{\faLanguage{}语言}
  \cvskill{中文}{5}
  \cvskill{英语}{4}
\end{minipage}
% Do not keep a blank line here.
\hfill%
% Do not keep a blank line here.
\begin{minipage}[t]{0.66\textwidth}

  \cvsection{\faClockO{}工作经历}

  \cvjob{高级软件工程师}{XXX科技有限公司}{2021.03 \textendash{} 现在}{四川\textbullet{}成都}
  \begin{itemize}
  \item 负责XXX的设计与开发
  \item 负责XXX的开发
  \end{itemize}
  
  \divider{}

  \cvjob{软件工程师}{XXX技术有限公司}{2019.12 \textendash{} 2021.02}{四川\textbullet{}成都}
  \begin{itemize}
  \item 负责XXX的开发
  \item 实现了XXX功能
  \end{itemize}

  \divider{}

  \cvjob{实习软件开发}{BBB技术有限公司}{2018.10 \textendash{} 2019.02}{四川\textbullet{}成都}
  \begin{itemize}
  \item 负责XXX的开发
  \item 完成了XXX工作
  \end{itemize}

  \cvsection{\faGithubAlt{}开源贡献}
  \cvopensource{XXX开源项目}
  \begin{itemize}
  \item 修复XXX的BUG
  \item 增加了XXX功能
  \end{itemize}

  \divider{}

  \cvopensource{XXX开源项目}
  \begin{itemize}
  \item 实现XXX功能
  \item 提交XXX的BUG
  \end{itemize}

\end{minipage}

\newpage

\cvsection{\faCubes{}项目经验}

\cvproject{XXX项目}{XXX技术有限公司}{Golang, Linux}{2021.06 \textendash{} 现在}
\begin{itemize}
\item 实现了XXX功能
\end{itemize}

\divider{}
\cvproject{XXX迁移项目}
          {XXX科技有限公司}
          {Oracle, PostgreSQL, ora2pg}
          {2021.03 \textendash{} 2021.06}
\begin{itemize}
\item 完成XXX迁移工作
\end{itemize}

\divider{}

\cvproject{基于 \LaTeX{} 的简历模版}
          {个人项目}
          {\LaTeX{}}{2021.10 \textendash{} 现在}
\begin{itemize}
\item 基于 \href{https://www.overleaf.com/articles/grant-morgans-first-latex-resume/jtdbtcctgnrr}{Grant Morgan} 和 \href{https://www.overleaf.com/articles/peter-rasmussens-resume-data-scientist/bphkfprrcnwv}{Peter Rasmussen} 的 \LaTeX{} 模版开发
\item 添加中文支持
\item 采用 Peter Rasmussen 简历模版中的方式对页面进行分栏
\item 增加 \verb|\cvjob|,\verb|\cvproject|,\verb|\cvopensource| 等命令来细化各个部分的输出
\end{itemize}

\end{document}
